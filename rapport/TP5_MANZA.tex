\documentclass[a4paper,12pt]{article}
\usepackage[utf8]{inputenc}
\usepackage[T1]{fontenc}
\usepackage[margin=1in]{geometry}
\usepackage{setspace}
\usepackage{titlesec}

% Title and author
\title{TP5 - Etude et amélioration d’une application}
\author{Jesimiel Manza}
\date{\today}

\begin{document}

\maketitle

\section*{Abstract}
% Your abstract goes here.

\section{Introduction}
Dans ce TP, nous allons nous interesser sur une application permettant de simuler la vie d'une colonie de fourmis peintres. Les fourmis se déplacent sur un surface sans bord.Le déplacement d’une fourmi obéit à des règles simples : soit elle détecte à proximité une couleur qui l’intéresse et décide de la suivre ou pas, soit elle se déplace aléatoirement. A chaque déplacement, la fourmi dépose sa couleur sur la surface. Sur une exécution longue, une auto-organisation apparaît. 
L'objectif de ce TP est d'analyser les performances de l'application existante et les faiblesse du code. Lorsque cette permière partie de vérification 

\section{Présentation de l'application}
% Your literature review goes here.

\section{Amélioration}
% Your methodology goes here.

\section{Analyse finale}
% Your results go here.

\section{Discussion}
% Your discussion goes here.

\section{Conclusion}
% Your conclusion goes here.

\section*{Acknowledgments}
% Your acknowledgments (if any) go here.

\begin{thebibliography}{9}
    % Your bibliography goes here.
\end{thebibliography}

\end{document}
